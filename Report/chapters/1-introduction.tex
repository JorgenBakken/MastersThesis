\chapter{Introduction}
\label{ch:introduction}

%%% Introduction to shape analysis

Humans begin learning geometry early by interacting with their environment and recognizing different shapes. This natural understanding evolves as they grow and may turn into formal mathematical knowledge. Over time, this intuitive sense of shapes and spatial relationships may be quantified, structured, and expanded into a comprehensive understanding of geometry.

The field of shape analysis traces its origins to D'Arcy Thompson \cite{thompsonGrowthFormProject}, who studied the shapes of plants and animals, demonstrating how non-linear transformations could standardize the shapes of different organisms. Since then, various approaches have been developed to tackle shape analysis. One significant contribution to the field is Kendall's work in 1984 \cite{kendallShapeManifoldsProcrustean1984}, where he articulated the concept of shape as the equivalence class in a quotient space and informally defined it as "what remains after accounting for variations due to translations, rotations, and dilatations". Additionally, the differential geometric perspective, outlined by Srivastava et al. \cite{srivastavaAdvancesDifferentialgeometricApproaches2012}, has provided a comprehensive framework for shape analysis.

%%% Recent advances in shape analysis

Historically, shape comparison involved identifying specific feature points or landmarks along the boundaries of objects \cite{kendallShapeManifoldsProcrustean1984}. Lately more literature regarding analysis of shapes as elements of infinite-dimensional Riemannian manifolds \cite{srivastavaShapeAnalysisElastic2011}, an area initially explored by Younes \cite{younesComputableElasticDistances1998}, has grown. For a greater overview, see \cite{bauerOverviewGeometriesShape2014, srivastavaFunctionalShapeData2016}. Today shape analysis can be use for a wide range of applications, such as protein structure comparison, medical image diagnosis, plant leaf classification, inverse obstacle scattering and more \cite{liuMathematicalFrameworkProtein2011, eckhardtElasticEnergyRegularization2019, lagaLandmarkfreeStatisticalAnalysis2014, glaunesLargeDeformationDiffeomorphic2008}.

%%% Rough paths and signatures

Signatures, introduced by K.-T. Chen in the context of smooth paths \cite{chenIteratedIntegralsExponential1954} and later expanded by Lyons into the framework of geometric rough paths \cite{lyonsDifferentialEquationsDriven1998}, have proven essential in diverse areas such as analyzing solutions to controlled differential equations, tackling classification challenges in time series, enhancing approaches in machine learning \cite{chevyrevPrimerSignatureMethod2016}, and in topological data analysis \cite{chevyrevPersistencePathsSignature2020}. Consequently, the theory of rough paths has developed into a comprehensive set of methods, serving both theoretical mathematical investigations and practical applications.

%%% Shape analysis on motion capture data

Shape analysis can address various problems in computer animation, such as periodicity, looping of movement, interpolation between movements, and enhancing motion recognition \cite{eslitzbichlerModellingCharacterMotions2015,celledoniShapeAnalysisLie2016,celledoniSignaturesShapeAnalysis2019,kovarAutomatedExtractionParameterization2004,pejsaStateArtExampleBased2010}. One technique for activity identification from video or motion capture data involves extracting silhouettes from video frames and applying shape comparison techniques, augmented by time-series modeling for analyzing sequences in shape spaces. Motion capture data, often provided in Acclaim Skeleton File (ASF) for skeleton structure and Acclaim Motion Capture (AMC) formats for motion data, captures an actor's movements onto a virtual skeleton for animating a 3D environment \cite{landerWorkingMotionCapture1998,adistambhaMotionClassificationUsing2008}.

A method for modeling these animations, represented by Eslitzbichler, involves treating them as curves on an \(n\)-torus, with each point on the curve corresponding to a pose in \(\mathbb{R}^3\) \cite{eslitzbichlerModellingCharacterMotions2015}. This approach facilitates movement comparison utilizes the Square Root Velocity Transform (SRVT) introduced in \cite{mioShapePlaneElastic2007, srivastavaShapeAnalysisElastic2011}. Celledoni et al. expanded on this concept by applying SRVT to curves valued in Lie groups, specifically modeling animations within \(SO(3)^n\) and later generalizing it to homogeneous spaces \cite{celledoniShapeAnalysisLie2016,celledoniShapeAnalysisHomogeneous2018}. Further, refining curve parametrization techniques, Bauer et al. introduced the use of gradient descent and dynamic programming algorithms to achieve optimal reparametrizations for comparing character animations \cite{bauerLandmarkGuidedElasticShape2015}. Building upon these methodologies, Celledoni et al. demonstrated the potential for classifying motion capture data using the framework from Bauer et al., alongside employing the logarithmic signature method \cite{celledoniSignaturesShapeAnalysis2019}. Additionally, Bærland et al. explored the use of deep neural networks for classification \cite{baerlandOptimalReparametrizationCurves2021}.

%%% How the thesis will contribute to the field

Srivastava et al. (2012) describe shape analysis through differential-geometric approaches as focusing on several primary objectives: quantifying shape differences, creating templates for specific shape classes, modeling shape variations through statistical models, and clustering, classifying, and estimating shapes \cite{srivastavaAdvancesDifferentialgeometricApproaches2012}. This thesis aims to address the quantification of shape differences, as well as the clustering, classification, and estimation of shapes, utilizing methodologies within Lie groups. This work explores reparameterization techniques, focusing on two methods: SRVT (Square Root Velocity Transform) combined with dynamic programming, and geodesic interpolation. These methods are utilized to achieve optimal curve alignment for comparing curves in Lie groups. Additionally, this work investigates the logarithmic signature method as an alternative approach to shape comparison in Lie groups. The comparison is based on assessing the geometric shape distance between curves. This work builds upon and expands the methodologies presented in \cite{celledoniSignaturesShapeAnalysis2019}.

%%% Structure of the thesis
The thesis begins by laying the groundwork in Chapter 2. This chapter introduces the fundamental theories essential for understanding the methods discussed later. Next, Chapter 3 delves into reparameterization. It examines two approaches: the Square Root Velocity Transform (SRVT) for adjusting curves in Lie groups through dynamic programming, and a geodesic interpolation method. Chapter 4 explores the logarithmic signature method as an alternative for comparing shapes within Lie groups. This method offers a less computationally expensive approach. Moving to real-world applications, Chapter 5 applies these methods to motion capture data. This chapter also discusses various ways to refine this data. Finally, Chapter 6 concludes the thesis by summarizing the key findings and suggesting directions for future research.

%%% Contribution to the UN SDGs

The research conducted in this thesis has the potential to contribute to several of the United Nations Sustainable Development Goals (SDGs) \cite{17GOALSSustainable}. The techniques developed are particularly relevant to SDG 3 (Good Health and Well-Being), as they facilitate the classification of human movements, which can enhance the development of personalized rehabilitation programs and medical treatments. Additionally, this research can advance SDG 9 (Industry, Innovation, and Infrastructure) by improving robotic movement classifications, thereby enhancing industrial efficiency, reducing waste, and improving safety across manufacturing sectors. Moreover, the precise analysis of animal movements contributes to SDG 14 (Life Below Water) and SDG 15 (Life on Land) by offering better conservation strategies and management practices through deeper insights into the behaviors and interactions of both marine and terrestrial species. By providing a comprehensive understanding of shape analysis, this research aligns with the UN’s broader goals for sustainable development.

\section*{Note to the Reader}

This thesis builds upon the foundational work presented in my project thesis \cite{bakkenLieGroupMethods2023} from fall 2023, which marked my initial exploration into the fields of topology and algebra. The implementation and code can be found in the authors GitHub repository\footnote{https://github.com/JorgenBakken/MastersThesis}.

In this document, various methodologies are employed to address analogous problems, with redundancy minimized by introducing synthetic data in the initial chapter and referencing it throughout. While expanding from individual elements to collections within Lie groups, we revisit key theoretical concepts to maintain structural coherence without unnecessary complexity. The hat map notation "\(\wedge\)" for both \(\mathrm{SO}(3)\) and \(\mathrm{SE}(3)\) is used, as it is standard in the literature, despite its potential for confusion. I hope readers will understand these choices.