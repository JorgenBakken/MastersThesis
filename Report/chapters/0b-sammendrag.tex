\chapter*{Sammendrag}

I denne avhandlingen utforsker vi to reparametriseringsteknikker: en fullstendig diskretisert metode som bruker kvadratrot-hastighetstransformasjon med dynamisk programmering, og geodetisk interpolasjonsmetode ved bruk av sekvensiell minstekvadraters programmering. Begge metodene ble evaluert for deres effektivitet i å oppnå optimale reparametriseringer og for å skille kurver basert på deres geometriske former gjennom deres formromsavstand. I tillegg bruker forskningen den logaritmiske signaturen for å beregne formromsavstander for å skille geometriske former, som et alternativ til reparametriseringsmetodene. Denne studien utvider disse metodene fra den spesielle ortogonale gruppen (\(\mathrm{SO}(3)^n\)), som tidligere har vært forsket på, til den spesielle euklidske gruppen (\(\mathrm{SE}(3)^n\)), med syntetiske data som validerer deres potensielle nytteverdi.

Vi utvider også vår forståelse av disse metodene for (\(\mathrm{SO}(3)^n\)), ved å innlemme etterbehandlingssteg som dimensjonsreduksjon gjennom hovedkomponentanalyse og klassisk multidimensjonal skalering, samt klynging for å forbedre dataens evne til å skille mellom forskjellige geometriske former. Disse teknikkene ble anvendt på bevegelsesfangstdata, og ga lovende resultater.