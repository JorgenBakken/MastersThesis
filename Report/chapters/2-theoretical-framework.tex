\chapter{Theoretical Framework}
\label{ch:theoretical-framework}

In this chapter, we lay the theoretical groundwork for the methodologies explored later in this thesis. We introduce fundamental concepts for Lie Groups, shape space, and tools for analyzing curves in \(\mathrm{SO}(3)\) and \(\mathrm{SE}(3)\). Additionally, we consider the direct products \(\mathrm{SO}(3)^n\) and \(\mathrm{SE}(3)^n\), which facilitate the analysis of multiple rotations or rigid body motions, aiding in the study of complex shape dynamics in higher dimensions.

\section{Fundamental Concepts}
\label{sec:fundamental-concepts}

We refer to \cite{leeIntroductionSmoothManifolds2012} for the conventions and notations used in this thesis. A Lie group \(G\) is a differentiable manifold with a smooth group structure, ensuring smooth multiplication and inversion operations. The corresponding Lie algebra, \(\mathfrak{g}\), is the tangent space at the identity element of \(G\) and features a Lie bracket \([\cdot, \cdot] : \mathfrak{g} \times \mathfrak{g} \rightarrow \mathfrak{g}\), which is bilinear, antisymmetric, and satisfies the Jacobi identity.

For a Lie group \(G\) and elements \(g, h \in G\), left and right translations are denoted as \(L_g(h) = gh\) and \(R_g(h) = hg\). The differential of the right translation by \(h\), \(T_gR_h\), maps from the tangent space at \(g\), \(T_gG\), to the tangent space at \(gh\), \(T_{gh}G\).

A Lie algebra-valued one-form on a Lie group \(G\) assigns, at each point \(g \in G\), a linear functional from the tangent space at \(g\), \(T_gG\), to the Lie algebra \(\mathfrak{g}\). Specifically, for any tangent vector \(X_g \in T_gG\), there exists a one-form \(\omega_g\) in the cotangent space \(T^*_gG\) that maps \(X_g\) to an element of \(\mathfrak{g}\).

The Cartesian product \(G \times G\) consists of all pairs \((g_1, g_2)\) with \(g_1, g_2 \in G\), operating under component-wise group actions. The corresponding Lie algebra is \(\mathfrak{g} \oplus \mathfrak{g}\), with the Lie bracket on \(G \times G\) defined as:
\begin{equation}
    [(g_1, g_2), (h_1, h_2)] = ([g_1, h_1], [g_2, h_2]),
\end{equation}
where \([g_1, h_1]\) and \([g_2, h_2]\) are the Lie brackets within \(\mathfrak{g}\).
\newpage
\section{The Maurer–Cartan Form and The Right Logarithmic Derivative}
\label{maurer-cartan-form-and-right-log-der}

The 
-Cartan form, denoted by \(\omega_g\), is a fundamental one-form \(\omega_g : T_gG \to \mathfrak{g}\) on the Lie group \(G\) as detailed in \cite[p.~373]{krieglConvenientSettingGlobal}. At any point \(g \in G\), it is defined via right translation by \(g^{-1}\), denoted as \(R_{g^{-1}}^*\), and is given by:

\begin{align}
    \omega_g : T_gG \rightarrow \mathfrak{g}, \nonumber \\
    \omega_g = T_g(R_{g^{-1}}) = (R_{g^{-1}})^*,
    \label{eq:maurer-cartan-form}
\end{align}
acting on any tangent vector \(X \in T_gG\) as:

\begin{equation}
    \omega_g(X) = (R_{g^{-1}})^*X \in \mathfrak{g}.
    \label{eq:maurer-cartan-form-on-element}
\end{equation}

This form is right-invariant, linking the Lie algebra \(\mathfrak{g}\) to the group's structure, as noted in \cite[p.~71]{olverEquivalenceInvariantsSymmetry1995}.

For a smooth curve \(c: I \to G\) with \(I \subset \mathbb{R}\) and \(c \in C^\infty\), the right logarithmic derivative \(\delta^r c\) is defined by \cite{krieglConvenientSettingGlobal} as:

\begin{equation}
    \delta^r c(t) = (\mathrm{R}_{c(t)^{-1}})^* \dot{c}(t),
\end{equation}
where \(\dot{c}(t)\) is the time derivative of the curve \(c\).

In the general linear group \(\mathrm{GL}(n)\), with matrix multiplication as the group operation, the Cartan form is defined as follows:

\begin{equation}
    (R_g)_* (\dot{c}(t)) = \frac{d}{dt}(R_g \circ c(t)) = \frac{d}{dt}(c(t) \cdot g) = \dot{c}(t) \cdot g,
    \label{eq:right-log-derivative-gl}
\end{equation}
where \(\cdot\) denotes matrix multiplication. Therefore, the right logarithmic derivative in \(\mathrm{GL}(n)\) is given by:

\begin{equation}
    \delta^r c(t) = (R_{c(t)^{-1}})_* (\dot{c}(t)) = \dot{c}(t) \cdot c(t)^{-1}.
    \label{eq:maurer-cartan-gl}
\end{equation}

This effectively maps the tangent vector \(\dot{c}(t)\) at \(c(t)\) to the tangent space at the identity of \(\mathrm{GL}(n)\).
\section{Shape Analysis on Lie Groups}
\label{sec:shape-analysis-on-Lie-groups}

Shape analysis is crucial for understanding the structure and diversity of geometric objects. Within Lie groups, it extends mathematical analysis to dynamic shapes. Shape spaces capture intrinsic geometry, focusing on essential features rather than specific parameterizations.

\newpage
\subsection{The Shape Space}
\label{subsec:shape-space}

Consider two curves \(c_1, c_2 \in \mathrm{Imm}(I, G)\), where \(\mathrm{Imm}(I, G)\) represents the space of immersions from the interval \(I = [0,1] \subset \mathbb{R}\) into a Lie group \(G\). The group of all orientation-preserving diffeomorphism of the interval \(I\), denoted \(\mathrm{Diff}^+(I)\), is defined as follows, detailed in \cite{celledoniSignaturesShapeAnalysis2019}:

\begin{equation}
    \mathrm{Diff}^+(I) := \{\varphi \in C^\infty(I, I) \mid \varphi'(t) > 0 \, \forall \, t \in I, \varphi(0) = 0, \varphi(1) = 1\}.
    \label{eq:diffeomorphism-group}
\end{equation}

Following \cite{celledoniShapeAnalysisLie2016}, curves \(c_1\) and \(c_2\) are considered equivalent, belonging to the same shape space, if there exists an orientation-preserving diffeomorphism \(\varphi \in \mathrm{Diff}^+(I)\) such that \(c_1 = c_2 \circ \varphi\). This equivalence establishes an equivalence class under the action of \(\mathrm{Diff}^+(I)\), thus defining the shape space. Formally, the shape space is the quotient space:

\begin{equation}
    \mathcal{S} := \mathcal{P} / \mathrm{Diff}^+(I),
    \label{eq:shape-space}
\end{equation}
where the space \(\mathcal{P}\) includes the immersions, encompassing all curves with non-vanishing first derivatives, and is defined by:

\begin{equation}
    \mathcal{P} := \mathrm{Imm}(I, G),
    \label{eq:parameterized-space}
\end{equation}

This construction implies that the shape space \(\mathcal{S}\) captures all curves up to orientation-preserving reparameterizations, highlighting the geometric properties of shapes over their specific parameterizations.

\subsection{Geodesic Distance in Shape Space}
\label{subsec:geodesic-distance}

Consider two curves, \(c_1\) and \(c_2\), which are elements of the space of immersions \(\mathrm{Imm}(I, G)\), where \(G\) is a finite-dimensional Lie group and \(I\) is the interval \([0,1]\). To measure distances between such immersions, we employ a metric \(d_{\mathcal{P}}\) defined on the space \(\mathcal{P}\) \eqref{eq:parameterized-space}. It is crucial for \(d_{\mathcal{P}}\) to reflect the intrinsic geometric properties of the curves, independent of their parameterization. This requirement is encapsulated by the property of reparameterization invariance:

\begin{equation}
    d_{\mathcal{P}}(c_1 \circ \varphi, c_2 \circ \varphi) = d_{\mathcal{P}}(c_1, c_2), \quad \forall \, \varphi \in \mathrm{Diff}^+(I).
    \label{eq:reparameterization-invariance}
\end{equation}

The shape space metric \(d_{\mathcal{S}}\) is then defined by minimizing over all reparameterizations:

\begin{equation}
    d_{\mathcal{S}}(c_1, c_2) := \inf_{\varphi \in \mathrm{Diff}^+(I)} d_{\mathcal{P}}(c_1, c_2 \circ \varphi).
    \label{eq:shape-space-metric}
\end{equation}

This definition effectively removes the influence of parameterization, focusing purely on the geometric shape of the curves.

By asserting that the metric \(d_{\mathcal{P}}\) is reparameterization invariant for any two curves and for any orientation-preserving diffeomorphism \(\varphi \in \mathrm{Diff}^+(I)\), the metric \(d_{\mathcal{S}}\) also inherits this invariance \cite[Lemma 3.4]{celledoniShapeAnalysisLie2016}.

Utilization of the Euclidean norm might lead to vanishing distances for non-identical curves, as highlighted by Michor and Mumford \cite{michorVanishingGeodesicDistance2004}. To address this limitation, a more robust choice involves Sobolev-type metrics based on the arc length derivative \cite{michorOverviewRiemannianMetrics2007a}, which incorporate derivatives of curves to capture more subtle geometric differences.

\subsection{Square Root Velocity Transform (SRVT)}
\label{subsec:square-root-velocity-transform}

Introduced by Srivastava et al. \cite{srivastavaShapeAnalysisElastic2011}, the Square Root Velocity Transform (SRVT) is a pivotal tool in shape analysis \cite{bauerConstructingReparametrizationInvariant2014, bauerOverviewGeometriesShape2014, bauerLandmarkGuidedElasticShape2015, celledoniShapeAnalysisHomogeneous2018, celledoniShapeAnalysisLie2016, celledoniSignaturesShapeAnalysis2019, schmedingIntroductionInfinitedimensionalDifferential2022, tumpachTemporalAlignmentHuman2023}. The SRVT transforms curves into a space where standard linear operations are meaningful, simplifying the complex problems of shape analysis and comparison.

The transformation for curves within a Lie group is defined as follows \cite{celledoniShapeAnalysisLie2016}:

\begin{equation}
    \begin{aligned}
        \mathcal{R}: \mathrm{Imm}(I, G) \rightarrow \left\{q \in C^\infty(I, \mathfrak{g}) \mid q(t) \neq 0  \, \forall \, t \in I\right\}, \\
        q(t) = \mathcal{R}(c)(t) := \frac{R^{-1}_{c(t)_*}(\dot c(t))}{\sqrt{\|R^{-1}_{c(t)_*}(\dot c(t))\|}},
    \end{aligned}
    \label{eq:SRVT}
\end{equation}
where \(\dot{c}(t)\) denotes the tangent vector of the curve at point \(c(t)\), and \(R^{-1}_{c(t)_*}\) is the differential of the inverse right translation by \(c(t)\). This operation transforms the tangent vector into the Lie algebra \(\mathfrak{g}\), and the division by the square root of its norm ensures the result is a unit vector, thereby reparameterizing the curve by its arc length.

As noted in \cite{celledoniShapeAnalysisLie2016}, SRVT is equivariant with respect to reparameterization, satisfying \(\mathcal{R}(c \circ \varphi) = \mathcal{R}(c) \circ \varphi \cdot \sqrt{\dot{\varphi}}\). Additionally, SRVT is translation invariant, meaning \(\mathcal{R}(R_g \cdot c ) = \mathcal{R}(c)\). This implies that the transformed representation of a curve remains unchanged under translation by an element \(g\) in the Lie group.

The translation invariance property implies that SRVT does not retain information about the initial position of the curve, leading to its non-injectivity. This characteristic highlights a fundamental aspect of SRVT: while it effectively captures the geometric essence of a curve, it abstracts away certain specific details like the starting point and orientation in space. 

\subsection{Pseudometric Based on SRVT}
\label{subsec:pseudometric-based-on-SRVT}

The pseudometric \(d_{\mathcal{P}}\) on the space of immersions \(\mathcal{P}\) is defined following the framework in \cite[Definition 3.7]{celledoniShapeAnalysisLie2016}:
\begin{equation}
    d_{\mathcal{P}}(c_0, c_1) := \sqrt{\int_I \|q_0(t) - q_1(t)\|^2 \, dt} = d_{L^2}(\mathcal{R}(c_0), \mathcal{R}(c_1)),
\end{equation}
where \(q_i := \mathcal{R}(c_i)\) for \(i = 0, 1\). This pseudometric, \(d_{\mathcal{P}}\), is invariant under reparameterization, as shown in \cite[Proposition 3.8]{celledoniShapeAnalysisLie2016}.

The subspace \(\mathcal{P}_*\) is defined as a closed submanifold of \(\mathcal{P}\):
\begin{equation}
    \mathcal{P}_* := \{c \in \mathrm{Imm}(I, G) : c(0) = e\},
\end{equation}
where \(e\) denotes the identity element of the Lie group \(G\). This subset represents immersions that start at the identity, formally expressed as \(\mathcal{P} \cap C^\infty(I, G)\). To adjust any smooth curve \(c: I \rightarrow G\) to start at the identity, the curve is right-translated by applying the inverse of its initial point \(c(0)^{-1}\) to every point along the curve:
\begin{equation}
    c(t) \mapsto c(t) \cdot c(0)^{-1} \quad \forall \, t \in [0,1].
\end{equation}

The pseudometric \(d_{\mathcal{P}_*}\) on \(\mathcal{P}_*\) utilizes SRVT to induce a metric from the \(L^2\)-metric on the tangent space \(C^\infty(I, \mathfrak{g} \setminus \{0\})\):
\begin{equation}
    d_{\mathcal{P}_*}(c_0, c_1) := 
    d_{L^2}(\mathcal{R}(c_0 \cdot c_0(0)^{-1}), \mathcal{R}(c_1 \cdot c_1(0)^{-1})),
\end{equation}
where \(q_i = \mathcal{R}(c_i)\) for \(i = 0, 1\).

According to \cite[Definition 3.10]{celledoniShapeAnalysisLie2016}, we can define the shape space \(\mathcal{S}_*\) as
\begin{equation}
    \mathcal{S}_* := \mathcal{P}_* / \mathrm{Diff}^+(I).
\end{equation}

Finally, the pseudometric \(d_{\mathcal{S}_*}\) on \(\mathcal{S}_*\) is given by
\begin{equation}
    d_{\mathcal{S}_*}(c_0, c_1) = \inf_{\varphi \in \mathrm{Diff}^+(I)} d_{\mathcal{P}_*}(\mathcal{R}(c_0), \mathcal{R}(c_1 \circ \varphi)),
    \label{eq:shape-space-metric-id}
\end{equation}
providing a geodesic distance on \(\mathcal{S}_*\), as established in \cite{bruverisGEODESICCOMPLETENESSSOBOLEV2014}.
  
\subsection{Geodesic Interpolation}
\label{subsec:geodesic-interpolation}

Geodesic interpolation within a Lie group \(G\) allows for smooth transitions between two elements \(c_0, c_1 \in G\). This process involves mapping the elements to the associated Lie algebra using the logarithmic map, performing linear interpolation within the Lie algebra, and then mapping the interpolated elements back to the Lie group via the exponential map \cite{shingelInterpolationSpecialOrthogonal2009, marthinsenInterpolationLieGroups1999}.

For two elements \(c_0\) and \(c_1\) in \(G\), the geodesic interpolation path \(\zeta(t)\), where \(t \in [0, 1]\), is defined as:
\begin{equation}
    \zeta(t) = \exp(t \cdot \log(c_1 c_0^{-1})) c_0,
    \label{eq:geodesic-interpolation}
\end{equation}
where \(\zeta(t)\) represents the shortest path between \(c_0\) and \(c_1\) in the Lie group. This approach ensures that the interpolation path is smooth and respects the intrinsic geometric structure of the Lie group.
\section{Curves in \texorpdfstring{\(\mathrm{SO}(3)\)}{\mathrm{SO}(3)}}
\label{chap:Curves-in-SO3}

The special orthogonal group \(\mathrm{SO}(3)\) consists of all \(3 \times 3\) rotation matrices, rigorously defined by the conditions:

\begin{equation}
    \mathrm{SO}(3) = \{ R \in \mathbb{R}^{3 \times 3} \mid R^T R = I_3, \, \det(R) = 1 \},
    \label{eq:SO3}
\end{equation}
where \(I_3\) represents the \(3 \times 3\) identity matrix and \(R^T\) denotes the transpose of \(R\). Matrices in \(\mathrm{SO}(3)\) characterize all proper rotations in \(\mathbb{R}^3\), defined by their orthogonality, unity determinant, and orientation preservation \cite{hallLieGroupsLie2015}.

Following \cite{gallegoCompactFormulaDerivative2015,celledoniLieGroupIntegrators2022}, the Lie algebra associated with \(\mathrm{SO}(3)\), denoted by \(\mathfrak{so}(3)\), is the tangent space at the identity matrix \(I_3\). It contains all skew-symmetric matrices infinitesimally close to \(I_3\).

%%% Hat and Vee Maps for SO(3) %%%

To establish an isomorphism between \(\mathfrak{so}(3)\) and \(\mathbb{R}^3\), the hat map is introduced. For a vector \(\omega = [\omega_1, \omega_2, \omega_3]^T \in \mathbb{R}^3\), we associate a skew-symmetric matrix in \(\mathfrak{so}(3)\) via the hat map:

\begin{equation}
    \begin{aligned}
        \wedge : \mathbb{R}^3 \rightarrow \mathfrak{so}(3), \\
        \hat{\omega} = \begin{bmatrix} \omega_1 \\ \omega_2 \\ \omega_3 \end{bmatrix}^\wedge
        =
        \begin{bmatrix} 0 & -\omega_3 & \omega_2 \\ \omega_3 & 0 & -\omega_1 \\ -\omega_2 & \omega_1 & 0 \end{bmatrix}. 
    \end{aligned}
    \label{eq:hat_SO3}
\end{equation}

The vee map performs the inverse operation:

\begin{equation}
    \begin{aligned}
        \vee : \mathfrak{so}(3) \rightarrow \mathbb{R}^3, \\
        \omega = \hat{\omega}^\vee =
        \begin{bmatrix} 0 & -\omega_3 & \omega_2 \\ \omega_3 & 0 & -\omega_1 \\ -\omega_2 & \omega_1 & 0 \end{bmatrix}^\vee
        =
        \begin{bmatrix} \omega_1 \\ \omega_2 \\ \omega_3 \end{bmatrix}.
    \end{aligned}
    \label{eq:vee_SO3}
\end{equation}
Consequently, for any \(\omega \in \mathbb{R}^3\), the identity \(\hat{\omega}^\vee = \omega\) holds true, affirming that the composition of the hat map followed by the vee map constitutes the identity mapping on \(\mathbb{R}^3\).

%%% Exponential and Logarithm Maps for SO(3) %%%

Rodrigues' rotation formula connects the matrix exponential map from the Lie algebra \(\mathfrak{so}(3)\) to the Lie group \(\mathrm{SO}(3)\) \cite{celledoniLieGroupMethods2003, maInvitation3DVision2004, gallegoCompactFormulaDerivative2015}:

\begin{equation}
    \begin{aligned}
        \exp : \mathfrak{so}(3) \rightarrow \mathrm{SO}(3), \\
        R = \exp(\hat{\omega}) = I_3 + \frac{\sin(\theta)}{\theta} \hat{\omega} + \frac{1 - \cos(\theta)}{\theta^2} \hat{\omega}^2,
    \end{aligned}
    \label{eq:exp_SO3}
\end{equation}
where \(\theta = \|\omega\|\) is the norm of the vector \(\omega\), representing the magnitude of rotation. This formula provides a direct method to convert a skew-symmetric matrix to a rotation matrix.

For a rotation matrix \(R \in \mathrm{SO}(3)\), the rotation angle \(\theta\) can be retrieved by:

\begin{equation}
    \theta = \arccos\left(\frac{\text{trace}(R) - 1}{2}\right),
\end{equation}
which is used in the computation of the matrix logarithm map \cite{shingelInterpolationSpecialOrthogonal2009, gallegoCompactFormulaDerivative2015}:

\begin{equation}
    \begin{aligned}
        \log : \mathrm{SO}(3) \rightarrow \mathfrak{so}(3), \\
        \hat{\omega} = \log(R) = \frac{\theta}{2 \sin(\theta)} (R - R^T),
    \end{aligned}
    \label{eq:log_SO3}
\end{equation}
where we assume that \(\theta\) is not a multiple of \(\pi\) to avoid a singularity in the denominator.

%%% Norm for SO(3) %%%

The Frobenius norm is used to measure the magnitude of elements in $\mathfrak{so}(3)$, allowing direct comparison with the Euclidean norm in $\mathbb{R}^3$. The norms are related by:

\begin{equation}
    \frac{1}{2} \|\hat{\omega}\|^2_F = \frac{1}{2}\text{trace}(\hat{\omega}^T \hat{\omega}) = \omega_1^2 + \omega_2^2 + \omega_3^2 = \|\omega\|^2. 
    \label{eq:norm_SO3}
\end{equation}

%%% Intro to SO(3)^d %%%

This framework can be extended to \(\mathrm{SO}(3)^d\), the Cartesian product of \(d\) copies of \(\mathrm{SO}(3)\), defined as:

\begin{equation}
    \mathrm{SO}(3)^d = \{ (R_1, R_2, \dots, R_d) \mid R_i \in \mathrm{SO}(3), \, i=1,\dots,d \},
\end{equation}
where each \(R_i\) is a \(3 \times 3\) rotation matrix. This definition ensures that each component \(R_i\) is an independent \(3 \times 3\) rotation matrix, operating within its own three-dimensional space. The collection of these matrices forms a product group that encapsulates rotational motions in \(d\) independent three-dimensional spaces.

The tangent space of the Cartesian product \(G \times G\) at the identity is isomorphic to the direct sum of the tangent spaces of \(G\) at each identity \cite{leeIntroductionSmoothManifolds2012}: 

\begin{equation}
    T_{(e,e)}(G \times G) \cong T_eG \oplus T_eG.
    \label{eq:tangent-space-product-group}
\end{equation}

Since \(\mathrm{SO}(3)^d\) is a Cartesian product of \(d\) copies of \(\mathrm{SO}(3)\), its tangent space at the identity element, \((e, e, \dots, e)\), where \(e\) denotes the \(3 \times 3\) identity matrix, is isomorphic to the direct sum of the tangent spaces of each \(\mathrm{SO}(3)\) at the identity:

\begin{equation}
    T_{(e, e, \dots, e)}(\mathrm{SO}(3)^d) \cong \bigoplus_{i=1}^d T_e \mathrm{SO}(3).
\end{equation}

The Lie algebra of a Lie group is isomorphic to its tangent space at the identity. Therefore, for \(\mathrm{SO}(3)^d\), the corresponding Lie algebra, denoted \(\mathfrak{so}(3)^d\), is the direct sum of \(d\) copies of \(\mathfrak{so}(3)\):

\begin{equation}
    \mathfrak{so}(3)^d := \bigoplus_{i=1}^d \mathfrak{so}(3).
\end{equation}
Operations within \(\mathfrak{so}(3)^d\) are performed component-wise, allowing for the independent manipulation of each component. This extension enables the application of the matrix exponential and logarithm maps to \(\mathfrak{so}(3)^d\) by applying these maps to each component separately. This enables us to extend the matrix exponential map and the matrix logarithmic map to \(\mathfrak{so}(3)^d\) by applying them to each component of the input matrix.

For an element \(\hat \omega^d = (\hat \omega_1, \hat \omega_2, \dots, \hat \omega_d) \in \mathfrak{so}(3)^d\), the matrix exponential map is defined as:

\begin{equation}
    \begin{aligned}
        \exp : \mathfrak{so}(3)^d \rightarrow \mathrm{SO}(3)^d, \\
        R^d = \exp(\hat \omega^d) = (\exp(\hat \omega_1), \exp(\hat \omega_2), \dots, \exp(\hat \omega_d)),
    \end{aligned}
    \label{eq:exp_SO3d}
\end{equation}
where \(\exp(\hat \omega_i)\) is the matrix exponential of the \(i\)-th component of \(\hat \omega^d\). The matrix logarithmic map is defined similarly:

\begin{equation}
    \begin{aligned}
        \log : \mathrm{SO}(3)^d \rightarrow \mathfrak{so}(3)^d, \\
        \hat \omega^d = \log(R^d) = (\log(R_1), \log(R_2), \dots, \log(R_d)),
    \end{aligned}
    \label{eq:log_SO3d}
\end{equation}
where \(\log(R_i)\) is the matrix logarithm of the \(i\)-th component of \(R^d\). These maps provide a direct method to convert skew-symmetric matrices to rotation matrices in \(\mathfrak{so}(3)^d\) and vice versa, facilitating independent operations on each component in a multi-body system.

%%% Hat and vee maps for SO(3)^d %%%


It is possible to extend the hat and vee maps, as defined in \eqref{eq:hat_SO3} and \eqref{eq:vee_SO3}, to \( \mathfrak{so}(3)^d \). The hat map is defined as:

\begin{equation}
    \begin{aligned}
        \wedge : \mathbb{R}^{3d} \rightarrow \mathfrak{so}(3)^d, \\
        \hat{\omega}^d = 
        \begin{bmatrix}
            \hat{\omega}_1 & 0 & \dots & 0 \\
            0 & \hat{\omega}_2 & \dots & 0 \\
            \vdots & \vdots & \ddots & \vdots \\
            0 & 0 & \dots & \hat{\omega}_d
        \end{bmatrix},
    \end{aligned}
    \label{eq:hat_SO3d}
\end{equation}
where \( \hat{\omega}_i \) is the skew-symmetric matrix associated with the \(i\)-th component of \( \omega^d \). The vee map is defined as:

\begin{equation}
    \begin{aligned}
        \vee : \mathfrak{so}(3)^d \rightarrow \mathbb{R}^{3d}, \\
        \omega^d = 
        \begin{bmatrix}
            \omega_{1,1} \\ \omega_{1,2} \\ \omega_{1,3} \\
            \vdots \\
            \omega_{d,1} \\ \omega_{d,2} \\ \omega_{d,3}
        \end{bmatrix},
    \end{aligned}
    \label{eq:vee_SO3d}
\end{equation}
where \( \omega_i \) is the vector associated with the \(i\)-th component of \( \hat{\omega}^d \). These maps allow for the conversion of vectors in \( \mathbb{R}^{3d} \) to skew-symmetric matrices in \( \mathfrak{so}(3)^d \) and vice versa. This facilitates the manipulation of multi-body systems in a vectorized form, enhancing computational efficiency and providing clarity in handling complex rotational dynamics.

%%% Norm for SO(3)^d %%%

For the Cartesian product, we extend the Frobenius norm to \(\mathfrak{so}(3)^d\) by summing the squared Frobenius norms of each component:

We define the norm on \(\hat{\omega}^d\) as:

\begin{equation}
    \|\hat{\omega}^d\|_F^2 
    := 
    \sum_{i=1}^d \|\hat{\omega}_i\|_F^2, 
\end{equation}
providing the following relation: 

\begin{equation}
    \frac{1}{2}\|\hat{\omega}^d\|_F^2 
    = 
    \frac{1}{2} \sum_{i=1}^d \text{trace}(\hat{\omega}_i^T \hat{\omega}_i)
    =
    \sum_{i=1}^d (\omega_{i,1}^2 + \omega_{i,2}^2 + \omega_{i,3}^2)
    =
    \|\omega^d\|_2^2, 
    \label{eq:norm_SO3d}
\end{equation}
which relates the Frobenius norm of a skew-symmetric matrix in \( \mathfrak{so}(3)^d \) to the Euclidean norm of the corresponding vector in \( \mathbb{R}^{3d} \). 

\section{Curves in \texorpdfstring{\(\mathrm{SE}(3)\)}{SE(3)}}
\label{sec:curves-in-SE3}

The special Euclidean group \(\mathrm{SE}(3)\) encapsulates all possible configurations of a rigid body in three-dimensional space, encompassing both rotations and translations. This group consists of \(4 \times 4\) matrices representing rigid transformations, defined as:

\begin{equation}
    \mathrm{SE}(3) = \left\{ T \in \mathbb{R}^{4 \times 4} \mid T = \begin{bmatrix} R & t \\ 0 & 1 \end{bmatrix}, R \in \mathrm{SO}(3), t \in \mathbb{R}^3 \right\},
    \label{eq:SE3}
\end{equation}
where \(R\) is a rotation matrix from the special orthogonal group \(\mathrm{SO}(3)\), and \(t\) is a translation vector in \(\mathbb{R}^3\). This ensures transformations preserve distances and angles, maintaining the rigid body's geometric integrity \cite{wangNonparametricSecondOrderTheory2008, blanco-claracoTutorialMathbfSE2022}.

The Lie algebra of \(\mathrm{SE}(3)\), denoted \(\mathfrak{se}(3)\), represents the tangent space at the identity element \(I_4\). It comprises \(4 \times 4\) matrices that are infinitesimally close to the identity, describing small motions of a rigid body \cite{wangNonparametricSecondOrderTheory2008, blanco-claracoTutorialMathbfSE2022}:

\begin{equation}
    \mathfrak{se}(3) = \left\{ \hat{\xi} \in \mathbb{R}^{4 \times 4} \mid \hat{\xi} = \begin{bmatrix} \hat{\omega} & v \\ 0 & 0 \end{bmatrix}, \omega \in \mathbb{R}^3, v \in \mathbb{R}^3 \right\},
\end{equation}
where \(\hat{\omega}\) is the skew-symmetric matrix of the angular velocity vector \(\omega\), and \(v\) represents the infinitesimal translations.

The hat (\(\wedge\)) and vee (\(\vee\)) maps facilitate the correspondence between \(\mathbb{R}^6\) and \(\mathfrak{se}(3)\) \cite{wangNonparametricSecondOrderTheory2008, blanco-claracoTutorialMathbfSE2022}, allowing transitions between vector and matrix representations:

\begin{equation}
    \begin{aligned}
        \wedge : \mathbb{R}^6 \rightarrow \mathfrak{se}(3), \\
        \hat{\xi} = \begin{bmatrix} \omega \\ v \end{bmatrix}^\wedge = \begin{bmatrix} \hat{\omega} & v \\ 0 & 0 \end{bmatrix} = \begin{bmatrix} 0 & -\omega_3 & \omega_2 & v_1 \\ \omega_3 & 0 & -\omega_1 & v_2 \\ -\omega_2 & \omega_1 & 0 & v_3 \\ 0 & 0 & 0 & 0 \end{bmatrix}.
    \end{aligned}
    \label{eq:hat_SE3}
\end{equation}

This operation converts a twist vector from \(\mathbb{R}^6\) to its matrix representation in \(\mathfrak{se}(3)\), capturing combined rotational and translational velocities. Conversely, the vee operation reverts a matrix in \(\mathfrak{se}(3)\) to its vector form:

\begin{equation}
    \begin{aligned}
        \vee : \mathfrak{se}(3) \rightarrow \mathbb{R}^6, \\
        \xi = \hat{\xi}^\vee = \begin{bmatrix} \omega \\ v \end{bmatrix} = \begin{bmatrix} \omega_1 \\ \omega_2 \\ \omega_3 \\ v_1 \\ v_2 \\ v_3 \end{bmatrix}.
    \end{aligned}
    \label{eq:vee_SE3}
\end{equation}

%%% Exponential and Logarithm Maps for SE(3) %%%

The relationship between the Lie algebra \(\mathfrak{se}(3)\) and the Lie group \(\mathrm{SE}(3)\) is mediated by the exponential and logarithmic maps.

The exponential map transfers elements from the Lie algebra \(\mathfrak{se}(3)\) to the Lie group \(\mathrm{SE}(3)\), using exponential coordinates \(\xi = (\omega, v)\) \cite{wangNonparametricSecondOrderTheory2008}

\begin{equation}
    \begin{aligned}
        \exp : \mathfrak{se}(3) \rightarrow \mathrm{SE}(3), \\
        T = \exp(\hat{\xi}) = \exp \left(\begin{bmatrix} \omega \\ v \end{bmatrix}^\wedge \right) = \begin{bmatrix} \exp(\hat{\omega}) & J_l(\omega) v \\ 0 & 1 \end{bmatrix},
    \end{aligned}
    \label{eq:exp_SE3}
\end{equation}
where \(\hat{\xi}\) is the twist element, and \(J_l(\omega)\) is the left Jacobian matrix, given by \cite{wangNonparametricSecondOrderTheory2008}:

\begin{equation}
    J_l(\omega) = I_3 + \frac{1 - \cos(\|\omega\|)}{\|\omega\|^2} \hat{\omega} + \frac{\|\omega\| - \sin(\|\omega\|)}{\|\omega\|^3} \hat{\omega}^2,
    \label{eq:left_Jacobian}
\end{equation}

The left Jacobian accounts for the nonlinearity in the rotational component of the motion. Conversely, the logarithmic map reverts elements from the Lie group \(\mathrm{SE}(3)\) back to the Lie algebra \(\mathfrak{se}(3)\), which is essential for analytical and computational purposes \cite{wangNonparametricSecondOrderTheory2008}:

\begin{equation}
    \begin{aligned}
        \log : \mathrm{SE}(3) \rightarrow \mathfrak{se}(3), \\
        \hat{\xi} = \log(T) = \log\left(\begin{bmatrix} R & t \\ 0 & 1 \end{bmatrix}\right) = \begin{bmatrix} \log(R) & J_l^{-1}(\omega) t \\ 0 & 0 \end{bmatrix},
    \end{aligned}
    \label{eq:log_SE3}
\end{equation}
where \(J_l^{-1}(\omega)\) is the inverse of the left Jacobian matrix, formulated as \cite{wangNonparametricSecondOrderTheory2008}:

\begin{equation}
    J_l^{-1}(\omega) = I_3 - \frac{1}{2} \hat{\omega} + \left(\frac{1}{\|\omega\|^2} - \frac{1 + \cos(\|\omega\|)}{2 \|\omega\| \sin(\|\omega\|)}\right) \hat{\omega}^2,
    \label{eq:left_Jacobian_inverse}
\end{equation}

When working with \(\mathrm{SE}(3)\), it is important to understand how the norm of the hat map corresponds to the vector. We can examine this through the Frobenius and Euclidean norms.

\begin{equation}
    \|\hat{\xi}\|_F^2 
    = \text{trace}(\hat{\xi}^T \hat{\xi}) 
    = \|\hat{\omega}\|_F^2 + \|v\|_F^2 
    = 2\|\omega\|_2^2 + \|v\|_2^2,
    \label{eq:norm_SE3}
\end{equation}
where \(\|\cdot\|_F\) denotes the Frobenius norm, and \(\|\cdot\|_2\) denotes the Euclidean norm. The \(\text{trace}(\cdot)\) operator takes the trace of a matrix. This equation shows that it is not possible to directly convert the Frobenius norm of \(\hat{\xi}\) to the Euclidean norm of \(\xi\). While in \(\mathrm{SO}(3)\), there is a direct relationship between the Frobenius norm of \(\hat{\omega}\) and the Euclidean norm of \(\omega\), specifically \(\frac{1}{2}\|\hat{\omega}\|_F^2 = \|\omega\|_2^2\) as seen in \eqref{eq:norm_SO3}. This means that we need to scale the rotational error by a factor of 2, when working with the vector form, to equalize the contributions of the rotational and translational errors.

%%% Extension to SE(3)^d %%%

The concept of the special Euclidean group \(\mathrm{SE}(3)\) can be generalized to higher dimensions by considering \(\mathrm{SE}(3)^d\), the Cartesian product of \(d\) copies of \(\mathrm{SE}(3)\). This extended group is defined as:

\begin{equation}
    \mathrm{SE}(3)^d = \{ (T_1, T_2, \dots, T_d) \mid T_i \in \mathrm{SE}(3), \, i = 1, \dots, d \},
\end{equation}
where each \( T_i \) is an independent \( 4 \times 4 \) rigid body transformation matrix. This configuration represents motions in multiple, independent three-dimensional spaces.

The tangent space of the Cartesian product \(G \times G\) at the identity is isomorphic to the direct sum of the tangent spaces of \(G\) at each identity, as seen in \eqref{eq:tangent-space-product-group}. Thus, for \(\mathrm{SE}(3)^d\), the tangent space at the identity element \((e, e, \dots, e)\), where \( e \) denotes the identity element of \(\mathrm{SE}(3)\), follows this principle:

\begin{equation}
    T_{(e, e, \dots, e)}(\mathrm{SE}(3)^d) \cong \bigoplus_{i=1}^d T_e \mathrm{SE}(3).
\end{equation}

The Lie algebra of a Lie group is isomorphic to its tangent space at the identity. Therefore, for \(\mathrm{SE}(3)^d\), the corresponding Lie algebra, denoted \(\mathfrak{se}(3)^d\), is the direct sum of \( d \) copies of \(\mathfrak{se}(3)\)

\begin{equation}
    \mathfrak{se}(3)^d := \bigoplus_{i=1}^d \mathfrak{se}(3),
\end{equation}
indicating that the Lie algebra \(\mathfrak{se}(3)^d\) is isomorphic to the direct sum of the tangent spaces of \(\mathrm{SE}(3)\) at each identity element. 

Within \(\mathfrak{se}(3)^d\), operations are performed component-wise, enabling independent manipulation of each transformation matrix. This framework extends the matrix exponential and logarithmic maps to \(\mathfrak{se}(3)^d\) by applying these operations to each matrix component separately.

For an element \(\hat \xi^d = (\hat \xi_1, \hat \xi_2, \dots, \hat \xi_d)\) in \(\mathfrak{se}(3)^d\), the matrix exponential map is expressed as:

\begin{equation}
\begin{aligned}
\exp : \mathfrak{se}(3)^d \rightarrow \mathrm{SE}(3)^d, \\
T^d = \exp(\hat \xi^d) = \left(\exp(\hat \xi_1), \exp(\hat \xi_2), \dots, \exp(\hat \xi_d)\right),
\end{aligned}
\end{equation}
where \(\exp(\hat \xi_i)\) is the matrix exponential of the \(i\)-th component. Similarly, the matrix logarithmic map is defined as:

\begin{equation}
\begin{aligned}
\log : \mathrm{SE}(3)^d \rightarrow \mathfrak{se}(3)^d, \\
\hat \xi^d = \log(T^d) = \left(\log(T_1), \log(T_2), \dots, \log(T_d)\right),
\end{aligned}
\end{equation}
where \(\log(T_i)\) is the matrix logarithm of the \(i\)-th component. 

%%% Hat and Vee Maps for SE(3)^d %%%

The hat and vee maps, fundamental in representing twists and their corresponding matrices in \(\mathrm{SE}(3)\), can be efficiently extended to \(\mathrm{SE}(3)^d\).

The hat map extension maps vectors from \(\mathbb{R}^{6d}\) into matrices within \(\mathfrak{se}(3)^d\), as follows:

\begin{equation}
    \begin{aligned}
    \wedge : \mathbb{R}^{6d} \rightarrow \mathfrak{se}(3)^d, \\
    \hat{\xi}^d = (\hat{\xi}_1, \hat{\xi}_2, \dots, \hat{\xi}_d),
    \end{aligned}
\end{equation}
where \(\hat{\xi}_i\) is the matrix form of the \(i\)-th component of the twist vector \(\xi^d\). This conversion expresses the collective rotational and translational velocities of multiple bodies as a single entity. Conversely, the vee map converts matrices from \(\mathfrak{se}(3)^d\) back into vectors in \(\mathbb{R}^{6d}\):

\begin{equation}
    \begin{aligned}
        \vee : \mathfrak{se}(3)^d \rightarrow \mathbb{R}^{6d}, \\
        \xi^d = (\xi_1, \xi_2, \dots, \xi_d),
    \end{aligned}
\end{equation}
where \(\xi_i\) is the vector form of the \(i\)-th matrix in \(\hat{\xi}^d\). 

These extended maps facilitate the manipulation of high-dimensional transformations in a vectorized form, enhancing computational efficiency and clarity. 

%%% Norm for SE(3)\(^d\) %%%

For the Cartesian product \(\mathrm{SE}(3)^d\), the Frobenius norm in \(\mathfrak{se}(3)^d\) is defined as:

\begin{equation}
    \|\hat{\xi}^d\|_F^2 = \sum_{i=1}^d \|\hat{\xi}_i\|_F^2.
\end{equation}

This corresponds to the Euclidean norm in \(\mathbb{R}^{6d}\) as follows:

\begin{equation}
    \|\hat{\xi}^d\|_F^2 
    = \sum_{i=1}^d \|\xi_i\|_2^2 
    = \sum_{i=1}^d \left(2\|\omega_i\|_2^2 + \|v_i\|_2^2\right)
    = 2 \|\omega^d\|_2^2 + \|v^d\|_2^2,
    \label{eq:norm_SE3d}
\end{equation}
where \(\xi_i \in \mathbb{R}^6\) is decomposed into angular velocity \(\omega_i \in \mathbb{R}^3\) and linear velocity \(v_i \in \mathbb{R}^3\). This indicates that, in vector form, the rotational error is scaled by a factor of 2 compared to the translational error. To equalize their contributions to the overall error, one would need to either double the weight of the rotational error or halve the weight of the translational error.   