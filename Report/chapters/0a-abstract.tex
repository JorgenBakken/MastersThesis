\chapter*{Abstract}

In this thesis, we explore two reparameterization techniques: a fully discretized method employing the Square Root Velocity Transform (SRVT) with dynamic programming, and a geodesic interpolation method using Sequential Least Squares Programming (SLSQP). Both methods were evaluated for their effectiveness in achieving optimal reparametrizations and for discretizing curves based on their geometric shapes through their shape space distance. Additionally, we utilize the logarithmic signature to compute shape space distances for distinguishing geometric forms, as an alternative to the reparameterization methods. This study extends these methods from the special orthogonal group (\(\mathrm{SO}(3)^n\)), previously researched, to the special Euclidean group (\(\mathrm{SE}(3)^n\)), with synthetic data validating their potential utility.

We further extend our understanding of these methods on (\(\mathrm{SO}(3)^n\)) by incorporating post-processing steps such as dimensionality reduction through Principal Component Analysis (PCA) and Classical Multi-Dimensional Scaling (cMDS), as well as clustering to enhance the data's ability to distinguish between different geometric forms. These techniques were applied to motion capture data, yielding promising results.