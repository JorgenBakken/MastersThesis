\section{Curves in \texorpdfstring{\(\mathrm{SE}(3)\)}{SE(3)}}
\label{sec:curves-in-SE3}

The special Euclidean group \(\mathrm{SE}(3)\) encapsulates all possible configurations of a rigid body in three-dimensional space, encompassing both rotations and translations. This group consists of \(4 \times 4\) matrices representing rigid transformations, defined as:

\begin{equation}
    \mathrm{SE}(3) = \left\{ T \in \mathbb{R}^{4 \times 4} \mid T = \begin{bmatrix} R & t \\ 0 & 1 \end{bmatrix}, R \in \mathrm{SO}(3), t \in \mathbb{R}^3 \right\},
    \label{eq:SE3}
\end{equation}
where \(R\) is a rotation matrix from the special orthogonal group \(\mathrm{SO}(3)\), and \(t\) is a translation vector in \(\mathbb{R}^3\). This ensures transformations preserve distances and angles, maintaining the rigid body's geometric integrity \cite{wangNonparametricSecondOrderTheory2008, blanco-claracoTutorialMathbfSE2022}.

The Lie algebra of \(\mathrm{SE}(3)\), denoted \(\mathfrak{se}(3)\), represents the tangent space at the identity element \(I_4\). It comprises \(4 \times 4\) matrices that are infinitesimally close to the identity, describing small motions of a rigid body \cite{wangNonparametricSecondOrderTheory2008, blanco-claracoTutorialMathbfSE2022}:

\begin{equation}
    \mathfrak{se}(3) = \left\{ \hat{\xi} \in \mathbb{R}^{4 \times 4} \mid \hat{\xi} = \begin{bmatrix} \hat{\omega} & v \\ 0 & 0 \end{bmatrix}, \omega \in \mathbb{R}^3, v \in \mathbb{R}^3 \right\},
\end{equation}
where \(\hat{\omega}\) is the skew-symmetric matrix of the angular velocity vector \(\omega\), and \(v\) represents the infinitesimal translations.

The hat (\(\wedge\)) and vee (\(\vee\)) maps facilitate the correspondence between \(\mathbb{R}^6\) and \(\mathfrak{se}(3)\) \cite{wangNonparametricSecondOrderTheory2008, blanco-claracoTutorialMathbfSE2022}, allowing transitions between vector and matrix representations:

\begin{equation}
    \begin{aligned}
        \wedge : \mathbb{R}^6 \rightarrow \mathfrak{se}(3), \\
        \hat{\xi} = \begin{bmatrix} \omega \\ v \end{bmatrix}^\wedge = \begin{bmatrix} \hat{\omega} & v \\ 0 & 0 \end{bmatrix} = \begin{bmatrix} 0 & -\omega_3 & \omega_2 & v_1 \\ \omega_3 & 0 & -\omega_1 & v_2 \\ -\omega_2 & \omega_1 & 0 & v_3 \\ 0 & 0 & 0 & 0 \end{bmatrix}.
    \end{aligned}
    \label{eq:hat_SE3}
\end{equation}

This operation converts a twist vector from \(\mathbb{R}^6\) to its matrix representation in \(\mathfrak{se}(3)\), capturing combined rotational and translational velocities. Conversely, the vee operation reverts a matrix in \(\mathfrak{se}(3)\) to its vector form:

\begin{equation}
    \begin{aligned}
        \vee : \mathfrak{se}(3) \rightarrow \mathbb{R}^6, \\
        \xi = \hat{\xi}^\vee = \begin{bmatrix} \omega \\ v \end{bmatrix} = \begin{bmatrix} \omega_1 \\ \omega_2 \\ \omega_3 \\ v_1 \\ v_2 \\ v_3 \end{bmatrix}.
    \end{aligned}
    \label{eq:vee_SE3}
\end{equation}

%%% Exponential and Logarithm Maps for SE(3) %%%

The relationship between the Lie algebra \(\mathfrak{se}(3)\) and the Lie group \(\mathrm{SE}(3)\) is mediated by the exponential and logarithmic maps.

The exponential map transfers elements from the Lie algebra \(\mathfrak{se}(3)\) to the Lie group \(\mathrm{SE}(3)\), using exponential coordinates \(\xi = (\omega, v)\) \cite{wangNonparametricSecondOrderTheory2008}

\begin{equation}
    \begin{aligned}
        \exp : \mathfrak{se}(3) \rightarrow \mathrm{SE}(3), \\
        T = \exp(\hat{\xi}) = \exp \left(\begin{bmatrix} \omega \\ v \end{bmatrix}^\wedge \right) = \begin{bmatrix} \exp(\hat{\omega}) & J_l(\omega) v \\ 0 & 1 \end{bmatrix},
    \end{aligned}
    \label{eq:exp_SE3}
\end{equation}
where \(\hat{\xi}\) is the twist element, and \(J_l(\omega)\) is the left Jacobian matrix, given by \cite{wangNonparametricSecondOrderTheory2008}:

\begin{equation}
    J_l(\omega) = I_3 + \frac{1 - \cos(\|\omega\|)}{\|\omega\|^2} \hat{\omega} + \frac{\|\omega\| - \sin(\|\omega\|)}{\|\omega\|^3} \hat{\omega}^2,
    \label{eq:left_Jacobian}
\end{equation}

The left Jacobian accounts for the nonlinearity in the rotational component of the motion. Conversely, the logarithmic map reverts elements from the Lie group \(\mathrm{SE}(3)\) back to the Lie algebra \(\mathfrak{se}(3)\), which is essential for analytical and computational purposes \cite{wangNonparametricSecondOrderTheory2008}:

\begin{equation}
    \begin{aligned}
        \log : \mathrm{SE}(3) \rightarrow \mathfrak{se}(3), \\
        \hat{\xi} = \log(T) = \log\left(\begin{bmatrix} R & t \\ 0 & 1 \end{bmatrix}\right) = \begin{bmatrix} \log(R) & J_l^{-1}(\omega) t \\ 0 & 0 \end{bmatrix},
    \end{aligned}
    \label{eq:log_SE3}
\end{equation}
where \(J_l^{-1}(\omega)\) is the inverse of the left Jacobian matrix, formulated as \cite{wangNonparametricSecondOrderTheory2008}:

\begin{equation}
    J_l^{-1}(\omega) = I_3 - \frac{1}{2} \hat{\omega} + \left(\frac{1}{\|\omega\|^2} - \frac{1 + \cos(\|\omega\|)}{2 \|\omega\| \sin(\|\omega\|)}\right) \hat{\omega}^2,
    \label{eq:left_Jacobian_inverse}
\end{equation}

When working with \(\mathrm{SE}(3)\), it is important to understand how the norm of the hat map corresponds to the vector. We can examine this through the Frobenius and Euclidean norms.

\begin{equation}
    \|\hat{\xi}\|_F^2 
    = \text{trace}(\hat{\xi}^T \hat{\xi}) 
    = \|\hat{\omega}\|_F^2 + \|v\|_F^2 
    = 2\|\omega\|_2^2 + \|v\|_2^2,
    \label{eq:norm_SE3}
\end{equation}
where \(\|\cdot\|_F\) denotes the Frobenius norm, and \(\|\cdot\|_2\) denotes the Euclidean norm. The \(\text{trace}(\cdot)\) operator takes the trace of a matrix. This equation shows that it is not possible to directly convert the Frobenius norm of \(\hat{\xi}\) to the Euclidean norm of \(\xi\). While in \(\mathrm{SO}(3)\), there is a direct relationship between the Frobenius norm of \(\hat{\omega}\) and the Euclidean norm of \(\omega\), specifically \(\frac{1}{2}\|\hat{\omega}\|_F^2 = \|\omega\|_2^2\) as seen in \eqref{eq:norm_SO3}. This means that we need to scale the rotational error by a factor of 2, when working with the vector form, to equalize the contributions of the rotational and translational errors.

%%% Extension to SE(3)^d %%%

The concept of the special Euclidean group \(\mathrm{SE}(3)\) can be generalized to higher dimensions by considering \(\mathrm{SE}(3)^d\), the Cartesian product of \(d\) copies of \(\mathrm{SE}(3)\). This extended group is defined as:

\begin{equation}
    \mathrm{SE}(3)^d = \{ (T_1, T_2, \dots, T_d) \mid T_i \in \mathrm{SE}(3), \, i = 1, \dots, d \},
\end{equation}
where each \( T_i \) is an independent \( 4 \times 4 \) rigid body transformation matrix. This configuration represents motions in multiple, independent three-dimensional spaces.

The tangent space of the Cartesian product \(G \times G\) at the identity is isomorphic to the direct sum of the tangent spaces of \(G\) at each identity, as seen in \eqref{eq:tangent-space-product-group}. Thus, for \(\mathrm{SE}(3)^d\), the tangent space at the identity element \((e, e, \dots, e)\), where \( e \) denotes the identity element of \(\mathrm{SE}(3)\), follows this principle:

\begin{equation}
    T_{(e, e, \dots, e)}(\mathrm{SE}(3)^d) \cong \bigoplus_{i=1}^d T_e \mathrm{SE}(3).
\end{equation}

The Lie algebra of a Lie group is isomorphic to its tangent space at the identity. Therefore, for \(\mathrm{SE}(3)^d\), the corresponding Lie algebra, denoted \(\mathfrak{se}(3)^d\), is the direct sum of \( d \) copies of \(\mathfrak{se}(3)\)

\begin{equation}
    \mathfrak{se}(3)^d := \bigoplus_{i=1}^d \mathfrak{se}(3),
\end{equation}
indicating that the Lie algebra \(\mathfrak{se}(3)^d\) is isomorphic to the direct sum of the tangent spaces of \(\mathrm{SE}(3)\) at each identity element. 

Within \(\mathfrak{se}(3)^d\), operations are performed component-wise, enabling independent manipulation of each transformation matrix. This framework extends the matrix exponential and logarithmic maps to \(\mathfrak{se}(3)^d\) by applying these operations to each matrix component separately.

For an element \(\hat \xi^d = (\hat \xi_1, \hat \xi_2, \dots, \hat \xi_d)\) in \(\mathfrak{se}(3)^d\), the matrix exponential map is expressed as:

\begin{equation}
\begin{aligned}
\exp : \mathfrak{se}(3)^d \rightarrow \mathrm{SE}(3)^d, \\
T^d = \exp(\hat \xi^d) = \left(\exp(\hat \xi_1), \exp(\hat \xi_2), \dots, \exp(\hat \xi_d)\right),
\end{aligned}
\end{equation}
where \(\exp(\hat \xi_i)\) is the matrix exponential of the \(i\)-th component. Similarly, the matrix logarithmic map is defined as:

\begin{equation}
\begin{aligned}
\log : \mathrm{SE}(3)^d \rightarrow \mathfrak{se}(3)^d, \\
\hat \xi^d = \log(T^d) = \left(\log(T_1), \log(T_2), \dots, \log(T_d)\right),
\end{aligned}
\end{equation}
where \(\log(T_i)\) is the matrix logarithm of the \(i\)-th component. 

%%% Hat and Vee Maps for SE(3)^d %%%

The hat and vee maps, fundamental in representing twists and their corresponding matrices in \(\mathrm{SE}(3)\), can be efficiently extended to \(\mathrm{SE}(3)^d\).

The hat map extension maps vectors from \(\mathbb{R}^{6d}\) into matrices within \(\mathfrak{se}(3)^d\), as follows:

\begin{equation}
    \begin{aligned}
    \wedge : \mathbb{R}^{6d} \rightarrow \mathfrak{se}(3)^d, \\
    \hat{\xi}^d = (\hat{\xi}_1, \hat{\xi}_2, \dots, \hat{\xi}_d),
    \end{aligned}
\end{equation}
where \(\hat{\xi}_i\) is the matrix form of the \(i\)-th component of the twist vector \(\xi^d\). This conversion expresses the collective rotational and translational velocities of multiple bodies as a single entity. Conversely, the vee map converts matrices from \(\mathfrak{se}(3)^d\) back into vectors in \(\mathbb{R}^{6d}\):

\begin{equation}
    \begin{aligned}
        \vee : \mathfrak{se}(3)^d \rightarrow \mathbb{R}^{6d}, \\
        \xi^d = (\xi_1, \xi_2, \dots, \xi_d),
    \end{aligned}
\end{equation}
where \(\xi_i\) is the vector form of the \(i\)-th matrix in \(\hat{\xi}^d\). 

These extended maps facilitate the manipulation of high-dimensional transformations in a vectorized form, enhancing computational efficiency and clarity. 

%%% Norm for SE(3)\(^d\) %%%

For the Cartesian product \(\mathrm{SE}(3)^d\), the Frobenius norm in \(\mathfrak{se}(3)^d\) is defined as:

\begin{equation}
    \|\hat{\xi}^d\|_F^2 = \sum_{i=1}^d \|\hat{\xi}_i\|_F^2.
\end{equation}

This corresponds to the Euclidean norm in \(\mathbb{R}^{6d}\) as follows:

\begin{equation}
    \|\hat{\xi}^d\|_F^2 
    = \sum_{i=1}^d \|\xi_i\|_2^2 
    = \sum_{i=1}^d \left(2\|\omega_i\|_2^2 + \|v_i\|_2^2\right)
    = 2 \|\omega^d\|_2^2 + \|v^d\|_2^2,
    \label{eq:norm_SE3d}
\end{equation}
where \(\xi_i \in \mathbb{R}^6\) is decomposed into angular velocity \(\omega_i \in \mathbb{R}^3\) and linear velocity \(v_i \in \mathbb{R}^3\). This indicates that, in vector form, the rotational error is scaled by a factor of 2 compared to the translational error. To equalize their contributions to the overall error, one would need to either double the weight of the rotational error or halve the weight of the translational error.